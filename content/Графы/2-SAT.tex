Выражение $a | b$ эквивалентно $(!a \rightarrow b) \& (!b \rightarrow a)$. Построим
граф импликаций: для каждой переменной в графе будет по две вершины, (обозначим
их через $x$ и $!x$), а рёбра в этом графе будут соответствовать импликациям.

\textbf{Правило.} Пусть $с[x]$ обозначает номер компоненты сильной связности,
которой принадлежит вершина $x$, причём номера упорядочены в порядке топологической
сортировки компонент сильной связности: если есть путь из $x$ в $y$, то
$c[x] \leq c[y]$.Тогда, если $c[x] > c[!x]$, то выбираем значение $x$, иначе выбираем
$!x$.

\begin { enumerate } \item Построим граф импликаций, заменив все выражения вида
$a | b$ двумя ребрами $!a \rightarrow b$ и $!b \rightarrow a$. \item Найдем все
компоненты сильной связности в графе импликаций. \item Проверим, что для любого
значения $x$ его отрицание лежит в другой компоненте сильной связности:
$c[x] \neq c[!x]$. Если это не так, то решения не существует. \item Если
требуется выводить ответ, то положим условие $x$ верным, если $c[x] < c[!x]$, и неверным
в противном случае. \end { enumerate }
